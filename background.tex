\chapter{Background}\label{C:background}

\section{Security System}
	\subsection{What defines a secure system?}
		Before someone can determine whether something is secure or not, we have to first create a baseline for what is a \emph{secure} system. In computer security, a typical approach is to require confidentiality, integrity, access control, availability, authentication and non-repudiation in the system\cite{Pfleeger:2006:SC:1177321}. \\

		These six attributes are described as follows:
		
		\begin{enumerate}
		
			\item \textbf{Confidentiality} is to ensure that secret information is never disclosed to unauthorized entities.

			\item \textbf{Access Control} is to ensure that authorized entities are granted permissions to the resource.
		
			\item \textbf{Integrity} is to ensure that data will not be corrupted.
	
			\item \textbf{Availability} is a guarantee of accessibility of data in the system.
			
			\item \textbf{Authentication} is the ability to verify the identity of an entity.
		
			\item \textbf{Non-repudiation} means that one cannot claim that certain actions were never performed. For example, if a message is transmitted or the act of signing the message.

		\end{enumerate}
		
		Although it is important to consider all these security attributes when designing a web application, its not always possible or necessarily required that all six attributes be fulfilled completely.
		
		\subsection{Cryptographic Basics}
			The basic aim of \emph{cryptography} is to enable two people to communicate over an insecure channel in a secure way. The term cryptography describes a range of cryptographic services including techniques for providing both confidentiality and authentication.


			\subsubsection{Symmetric Encryption}
			
			Symmetric Encryption, has a single secret key that it can use to encrypt plaintext. Then using the same secret key another user can decrypt it \cite{ferguson2003practical}. Symmetric key encryption is an essential mechanism for protecting data at rest. We can use this to reduce the risk of unauthorized access to sensitive data. However, the use of symmetric key encryption brings with it certain dangers. Most important is that, once encrypted, we need a robust mechanism to ensure the encryption key is protected from unauthorized access. The key management system needs to grant access to trust worthy entities, and restrict unauthorized entities to ensure that the key is secure.

			\subsubsection{Asymmetric Cryptography}
			
			Asymmetric Cryptography is a cryptographic algorithm which requires two separate keys, one of which is private and one which is public \cite{ferguson2003practical}.

			The public and private key pair comprise of two uniquely related cryptographic keys. The public key is made available to everyone via a publicly accessible repository or directory. While on the other hand the private key must remain \emph{confidential} to its respective owner. Because the key pair is mathematically related, the public key is used to encrypt the \emph{plaintext}, whereas the private key is used to decrypt the \emph{ciphertext} produced by the public key. Therefore, Asymmetric cryptography can provide confidentiality, as the user with the private key is only one able to decrypt the message \cite{pub-key}.

			\subsection{Basic Key Management}
			Encryption Key Management is the administration of tasks involved with protecting, storing, backing up and organizing encryption keys or secrets. \cite{defination:key-management}. It considers the general management of cryptographic keys, and the means by which public keys are distributed. The study of key management can be broken down into phases concerning the life-cycle of a cryptographic key. These four phases are: \\
			
			\begin{enumerate}
				\item \textbf{Key Generation} which covers the creation of the keys
				
				\item \textbf{Key Establishment} which is the methods by which the keys are distributed to the relevant users in the network.
				
				\item \textbf{Key Update} which the techniques used to renew or refresh the keys in the system
				
				\item \textbf{Key Destruction} which covers the deletion and disposal of keys when they are no longer of use.
			\end{enumerate}

			Establishing and distributing keys over an unsecured channel is an important consideration to make. An elegant and widely-applied scheme for establishing a shared key across an insecure channel is the \emph{Diffie-Hellman} key exchange. The Diffie-Hellman key exchange is a specific method of exchanging cryptographic keys. The Diffie-Hellman key exchange method allows two parties that have no prior knowledge of each other to jointly establish a shared secret key over an insecure communication channel. \\

	\subsection{Trust}

	\subsection{Authentication}

	\subsection{Authorization}

	\subsection{Access Control}\

\section{Key Management}
	\subsection{What is Key Management?}

	\subsection{Importance of Key Management}

	\subsection{Public Key Infrastructures}

\section{Defining the Cloud}
	\subsection{Cloud Service Models}

	\subsection{Security in the Cloud}

\section{Vendors}
	\subsection{Amazon Web Services}

	\subsection{Luna SA}

	\subsection{DNSSEC}
